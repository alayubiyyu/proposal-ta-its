\chapter{PENDAHULUAN}

\section{Latar Belakang}

% Ubah paragraf-paragraf berikut sesuai dengan LATAR BELAKANG dari tugas akhir
Saat ini metode penentuan posisi suatu titik di permukaan bumi mengalami kemajuan teknologi.
Hal tersebut ditandai dengan ketersediaan peralatan alat ukur yang dilengkapi dengan teknologi digital terkini.
Salah satu metode penentuan posisi suatu titik di permukaan bumi yaitu, dengan menggunakan \emph{Global Position System} (GPS).
Nama formalnya adalah NAVstar GPS, kependekan dari \emph{"NAVigation Satellite Timing and Ranging Global Positioning System"} \parencite{Abidin2007}.
GPS secara garis besar terdiri dari dua tipe yaitu tipe navigasi dan tipe geodetik. Perangkat GPS mengandalkan penerimaan sinyal dari setidaknya empat satelit.
Jika mereka terhubung hanya pada tiga satelit, maka posisinya tidak sepenuhnya akurat. 
Massalah dapat terjadi ketika rintangan seperti dinding, bangunan, gedung pencakar langit dan pepohonan yang menghalangi sinyal.
Kondisi atmosfer yang ekstrem seperti badai geomagnetik juga dapat menyebabkan masalah. Selain itu, teknologi pemetaan yang digunakan berrsama dengan \emph{Global Position System}
mungkin tidak akurat dan menyebabkan kesalahan dalam bernavigasi.

\emph{Dead Reckoning} secara bertahap mengintegrasi jarak tempuh dan arah perjalanan relatif ke lokasi awal yang diketahui.
Arah kendaraan biasanya ditentukan oleh kompas magnetik, dan jarak yang ditempuh dihitung berdasarkan waktu tempuh dengan kecepatan kendaraan. 
Namun, dalam navigasi berbasis darat modern, berbagai perangkat sensor dapat digunakan seperti perhitungan perputaran roda, giroskop, dan unit pengukuran inersia (IMU).
Kelemahan umum dari \emph{Dead Reckoning} bahwa kesalahan estimasi meningkat dengan jarak ke posisi awal yang diketahui, sehingga diperlukan pembaruan yang sering dengan posisi tetap.
Biasanya \emph{Dead Reckoning} dapat digunakan sebagai back-up jika sistem navigasi utama tidak tersedia atau tidak dapat digunakan.
Dengan menggunakan \emph{Deep Learning}, sistem \emph{Dead Reckoning} dapat belajar dari data yang telah dikumpulkan sebelumnya dan menggunakannya untuk memprediksi posisi saat ini dengan lebih akurat.
Diusulkan penelitian yang berjudul “Pengembangan Dead-Reckoning Berbasis Deep Learning Untuk Navigasi Jalan Dengan Sensor Berbiaya Rendah dan Mikrokontroler STM”.

\section{Rumusan Masalah}

% Ubah paragraf berikut sesuai dengan RUMUSAN MASALAH dari tugas akhir
Berdasarkan latar belakang diatas, penulis dapat merumuskan beberapa masalah penting sebagai berikut,

1. Bagaimana cara mengatasi \emph{Pedestrian Navigation} pada hutan dan goa yang tidak bisa menggunakan \emph{Global Position System}?

2. Bagaimana penerapan penggunaan \emph{Deep Learning} pada \emph{Computer-on-a-Chip}?

3. \emph{Noise Drifting} di sensor \emph{Inertial Measurement Unit} (IMU) yang tergolong masih besar.

\section{Batasan Masalah atau Ruang Lingkup}

% Ubah paragraf berikut sesuai dengan BATASAN MASALAH dari tugas akhir
Supaya memperoleh hasil yang maksimal mengenai masalah yang ada dalam penelitian dan mengingat keterbatasan yang ada juga,
maka penulis akan memberikan batasan sebagai berikut,

1. Pengumpulan data diruang terbuka, ruang lingkup disekitar kawasan integritas Institut Teknologi Sepuluh Nopember, Surabaya, Jawa Timur.

2. Menggunakan orde \emph{International System of Units} (SI) satuan Meter pada perhitungan panjang atau jarak.

3. Alat dikalibrasi dengan \emph{Global Position System} saat digunakan.

4. Navigasi dilakukan pada bidang Dua Dimensi (2D).

\section{Tujuan}

% Ubah paragraf berikut sesuai dengan TUJUAN PENELITIAN dari tugas akhir
Tujuan sejalan dengan rumusan masalah diatas, laporan ini disusun dengan tujuan untuk mendeskripsikan:

1. Untuk membuat sistem navigasi untuk di hutan dan goa berbasis \emph{Inertial Measurement Unit} (IMU).

2. Untuk membandingkan hasil tingkat akurasi dari penggunaan \emph{Global Position System} (GPS) dengan 
gabungan penggunaan metode \emph{Dead Reckoning}.

3. Untuk membuat model pengurang noise untuk \emph{Inertial Measurement Unit} (IMU).

\section{Manfaat}

% Ubah paragraf berikut sesuai dengan MANFAAT dari tugas akhir
Secara teoretis penelitian ini berguna sebagai pengembangan konsep materi atau ilmu dari beberapa mata kuliah yang didapatkan selama perkuliahan, 
dan Secara praktis laporan tugas akhir ini diharapkan dapat bermanfaat seperti,

1. Meningkatkan kebermanfaatan mikrokontroller di Indonesia.

2. Memberikan pengetahuan lebih kepada masyarakat luas tentang manfaat \emph{Dead Reckoning}.

3. Meminimalisasi tingkat tersesat yang ditimbulkan akibat pemakaian 
\emph{Global Position System} (GPS) pada lokasi susah sinyal.
