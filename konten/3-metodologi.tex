\chapter{METODOLOGI}

% Ubah konten-konten berikut sesuai dengan isi dari metodologi

\section{Metode yang digunakan}

% Ubah paragraf berikut sesuai dengan METODE YANG DIGUNAKAN dari tugas akhir
% Contoh input gambar dengan format *.jpg
\begin{figure} [ht] \centering
  % Nama dari file gambar yang diinputkan
  \includegraphics[scale=0.35]{gambar/RNN-3Dimensi.png}
  % Keterangan gambar yang diinputkan
  \caption{Pemrosesan data pada RNN untuk menghasilkan jarak yang ditempuh}
  % Label referensi dari gambar yang diinputkan
  \label{fig:RNN-3D}
\end{figure}

Dengan menerapkan sistem \emph{Pose and Position} dalam dua dimensi maka dari itu data yang dihasilkan pada Az dan Gz 
tidak diperlukan karena jika dipakai dibutuhkan setidaknya tiga dimensi pada objek target. Bisa diterapkan seperti 
gambar \ref{fig:Koef-Relatif} berikut ini:

% Contoh input gambar dengan format *.jpg
\begin{figure} [ht] \centering
  % Nama dari file gambar yang diinputkan
  \includegraphics[scale=0.55]{gambar/IMU-to-KRel-KMut.png}
  % Keterangan gambar yang diinputkan
  \caption{Flowchart menentukan Koef. Relatif dengan RNN}
  % Label referensi dari gambar yang diinputkan
  \label{fig:Koef-Relatif}
\end{figure}


% Contoh penggunaan referensi dari gambar yang diinputkan
Pada \emph{flowchart} yang tertera di Gambar \ref{fig:Koef-Relatif}.  fff

\section{Bahan dan peralatan yang digunakan}

\subsection{Mikrokontroller STM32}
% Ubah paragraf berikut sesuai dengan Mikrokontroller STM32 dari tugas akhir
Mikrokontroler STM adalah rangkaian mikrokontroler yang dikembangkan dan diproduksi oleh STMicroelectronics, sebuah perusahaan semikonduktor global yang berbasis di Eropa. 
Mikrokontroler STM banyak digunakan dalam berbagai aplikasi termasuk elektronik konsumen, otomotif, kontrol industri, dan sistem komunikasi. Biasanya diprogram menggunakan 
bahasa pemrograman tingkat tinggi seperti C atau C++, dan dapat diprogram menggunakan berbagai alat pengembangan dan lingkungan pengembangan terintegrasi (IDE). 
Mikrokontroller ini juga menyediakan serangkaian pustaka dan alat perangkat lunak untuk membantu pengembang membuat aplikasi untuk mikrokontroler mereka. Seri STM32 didasarkan pada inti ARM Cortex-M3 yang dirancang khusus untuk aplikasi tertanam yang membutuhkan kinerja tinggi, biaya rendah, dan konsumsi daya rendah. Ini dibagi menjadi 
produk yang berbeda sesuai dengan arsitektur inti: Di antara mereka, seri STM32F meliputi: seri "ditingkatkan" STM32F103, seri "dasar" STM32F101, STM32F105, seri "interkoneksi" STM32F107, 
dan seri yang ditingkatkan dengan frekuensi clock 72MHz, yang merupakan produk kinerja tertinggi di antara produk serupa. 

% Contoh input gambar dengan format *.jpg
\begin{figure} [ht] \centering
  % Nama dari file gambar yang diinputkan
  \includegraphics[scale=0.35]{gambar/STM32H750VBT6_Generic_Development_Board.jpg}
  % Keterangan gambar yang diinputkan
  \caption{ST-Microelectronics STM32H750VBT6, Arm Cortex-M7}
  % Label referensi dari gambar yang diinputkan
  \label{fig:STM-Board}
\end{figure}

\subsection{IMU Sensor Module MPU-9255}
% Ubah paragraf berikut sesuai dengan SENSOR MODULE IMU dari tugas akhir
MPU-9255 adalah \emph{Inertial Measurement Unit} (IMU) 9-sumbu yang dikembangkan oleh InvenSense, sebuah perusahaan yang berspesialisasi dalam desain dan pembuatan pelacakan gerak dan sensor 
pencitraan. MPU-9255 adalah perangkat kecil berdaya rendah yang dapat mengukur akselerasi, laju sudut, dan medan magnet dalam tiga dimensi. Ini umumnya digunakan dalam aplikasi seperti 
drone, robotika, realitas virtual, dan teknologi yang dapat dikenakan untuk memberikan informasi waktu nyata tentang orientasi, gerakan, dan akselerasi perangkat. 
Untuk menggunakan MPU-9255, Anda harus menghubungkannya ke mikrokontroler atau perangkat lain yang dapat berkomunikasi dengannya menggunakan antarmuka digital seperti I2C atau SPI. 
Nantinya kemudian perlu menulis kode untuk membaca data dari MPU-9255 dan menginterpretasikannya berdasarkan persyaratan aplikasi spesifik. InvenSense menyediakan serangkaian 
pustaka dan alat perangkat lunak untuk membantu pengembang membuat aplikasi menggunakan MPU-9255.

% Ubah paragraf berikut sesuai dengan PC/HP dari tugas akhir
\section{Urutan pelaksanaan penelitian}

% Ubah tabel berikut sesuai dengan isi dari rencana kerja
\newcommand{\w}{}
\newcommand{\G}{\cellcolor{gray}}
\begin{table}[h!]
  \begin{tabular}{|p{3.5cm}|c|c|c|c|c|c|c|c|c|c|c|c|c|c|c|c|}

    \hline
    \multirow{2}{*}{Kegiatan} & \multicolumn{16}{|c|}{Minggu} \\
    \cline{2-17} &
    1 & 2 & 3 & 4 & 5 & 6 & 7 & 8 & 9 & 10 & 11 & 12 & 13 & 14 & 15 & 16 \\
    \hline

    % Gunakan \G untuk mengisi sel dan \w untuk mengosongkan sel
    Pengambilan data &
    \G & \G & \G & \G & \w & \w & \w & \w & \w & \w & \w & \w & \w & \w & \w & \w \\
    \hline

    Pengolahan data &
    \w & \w & \w & \w & \G & \G & \G & \G & \w & \w & \w & \w & \w & \w & \w & \w \\
    \hline

    Analisa data &
    \w & \w & \w & \w & \w & \w & \w & \w & \G & \G & \G & \G & \w & \w & \w & \w \\
    \hline

    Evaluasi penelitian &
    \w & \w & \w & \w & \w & \w & \w & \w & \w & \w & \w & \w & \G & \G & \G & \G \\
    \hline

  \end{tabular}
  \captionof{table}{Tabel timeline}
  \label{tbl:timeline}
\end{table}

Pada \emph{timeline} yang tertera di Tabel \ref{tbl:timeline} \lipsum[10]
