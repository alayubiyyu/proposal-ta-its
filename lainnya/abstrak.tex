\begin{center}
  \large
  \textbf{PENGEMBANGAN \emph{DEAD RECKONING} BERBASIS \emph{DEEP LEARNING}  
  UNTUK NAVIGASI JALAN DENGAN SENSOR BERBIAYA RENDAH IMU 
  DAN MIKROKONTROLER STM}
\end{center}
\addcontentsline{toc}{chapter}{ABSTRAK}
% Menyembunyikan nomor halaman
\thispagestyle{empty}

\begin{flushleft}
  \setlength{\tabcolsep}{0pt}
  \bfseries
  \begin{tabular}{ll@{\hspace{6pt}}l}
  Nama Mahasiswa / NRP&:& Muhammad Rafie Azmi / 0721194000077\\
  Departemen&:& Teknik Komputer FTEIC - ITS\\
  Dosen Pembimbing&:& 1. Dion Hayu Fandiantoro, S.T., M.T.\\
  & & 2. Arief Kurniawan, S.T., M.T.\\
  \end{tabular}
  \vspace{4ex}
\end{flushleft}
\textbf{Abstrak}

% Isi Abstrak
Pengguna yang berpindah dari titik P0 ke titik P1 dapat diilustrasikan berada dalam elips keper- cayaan 95 persen berpusat pada P1 dengan sumbu AB, ditentukan oleh heading akurasi sensor, 
dan CD, ditentukan oleh sensor kecepatan ketepatan. Melakukan Dead Reckoning menggunakan Recurrent Neural Network, jaringan perlu dilatih pada kumpulan data input yang terdiri dari posisi 
kendaraan sebelumnya dan data sensorik yang dikumpulkan selama periode waktu tertentu, dan label output yang sesuai yang mewakili posisi kendaraan saat ini. Ada banyak aplikasi potensial 
untuk menggunakan Convolutional Neural Network untuk perhitungan mati, termasuk kendaraan otonom, drone, dan robot seluler lainnya. Kemampuan untuk secara akurat memperkirakan posisi 
kendaraan saat ini berdasarkan posisi sebelumnya dan data sensorik dapat sangat penting untuk navigasi dan lokalisasi di lingkungan di mana sinyal GPS. Dari penelitian yang akan dilakukan, 
diharapkan pengendara yang sudah dilengkkapi alat dapat menentukan sebuah posisi mereka tanpa perlu mengkhawatirkan medan perlitasan yang dilewati (seperti underground and forest) dengan 
penerapan metode Dead Reckoning berbasis Deep Learning sehingga menghasilkan luaran berupa navigasi secara Real Time dengan akurasi tinggi.



\vspace{2ex}
\noindent
\textbf{Kata Kunci: \emph{Dead Reckoning, RNN, Global Position System}}