\begin{center}
  \large
  \textbf{DEVELOPMENT OF DEEP LEARNING-BASED DEAD RECKONING 
  FOR ROAD NAVIGATION 
  WITH IMU LOW-COST SENSORS AND STM MICROCONTROLLERS}
\end{center}
% Menyembunyikan nomor halaman
\thispagestyle{empty}

\begin{flushleft}
  \setlength{\tabcolsep}{0pt}
  \bfseries
  \begin{tabular}{lc@{\hspace{6pt}}l}
  Student Name / NRP&: &Muhammad Rafie Azmi / 07211940000077\\
  Department&: &Computer Engineering FTEIC - ITS\\
  Advisor&: &1. Dion Hayu Fandiantoro, S.T., M.T.\\
  & & 2. Arief Kurniawan, S.T., M.T.\\
  \end{tabular}
  \vspace{4ex}
\end{flushleft}
\textbf{Abstract}

% Isi Abstrak
Users moving from point P0 to point P1 can be illustrated to be in an ellipse
95 percent confidence centered on P1 with AB axis, determined by heading accuracy
sensors, and CDs, are determined by the accuracy speed sensor.Doing Dead Reckoning
using the Recurrent Neural Network, the network needs to be trained on an input data set that
consists of the vehicle's previous position and sensory data collected during the period
specific time, and corresponding output labels that represent the current position of the vehicle. Exist
many potential applications for using Convolutional Neural Network for perhitu-
ngan die, including autonomous vehicles, drones and other mobile robots.Ability to
accurately estimate the current position of the vehicle based on its previous position and data
Sensory can be critical for navigation and localization in environments where GPS signals.From the research that will be carried out, it is expected that motorists who have been equipped with da-
Pat determines their position without worrying about the terrain of the perlitasan
bypassed (such as underground and forest) with the application of the Dead Reckoning-based method
Deep Learning so as to produce outputs in the form of navigation in Real Time with accuracy
tall.

\vspace{2ex}
\noindent
\textbf{Keywords: \emph{Dead Reckoning, RNN, Global Position System}}